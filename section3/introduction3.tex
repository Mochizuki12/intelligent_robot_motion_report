%!TEX root = ../thesis.tex

\section{第3章の導入}

経路計画には実際に選択されたパスは通れるのかという問題が残った.
これを解決するために,軌道生成を行う.
今回の軌道生成では,関節空間を用いて位置xの1自由度の対向二輪ロボットで行う.
はじめに,連続性の必要性について説明し,次にbang-bang制御を用いた軌道生成を解説する.
最後に,台形速度曲線の制御を解説する.

軌道生成で用いられるグラフは時間をパラメータとした位置・姿勢などの変化のグラフである.

\newpage
