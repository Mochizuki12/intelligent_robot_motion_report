%!TEX root = ../thesis.tex

\subsection{連続性のない制御}

\begin{figure}[H]
  \centering
 \includegraphics[keepaspectratio, scale=0.8]
      {images/png/inpulse.drawio.png}
 \caption{No continuity}
 \label{Fig:Step}
\end{figure}

ロボットの位置を変化させるには速度を出す必要がある.
初速度0m/sから目標速度までステップ状に変化させると時間に対する位置と加速度のグラフはFig\ref{Fig:Step}のようになる.
図を見ると加速度が無限大のインパルスを必要としていることがわかり,これは現実的ではない.
加速度が無限大のインパルスを必要としているのは,速度の変化がステップ状になっているからである.
すなわち,速度の変化量が無限になるからである.
ここからわかることは,速度は連続性が必要であるということである.
また,速度は位置の変化量であるため,位置も連続性が必要である.

このように速度に連続性のない制御は現実的ではないことがわかる.
逆に,加速度の連続性は必須ではないが加速度に1msぐらいの制御周期で連続性をもたせると,
なめらかな力制御を実現できる.

\newpage
