%!TEX root = ../thesis.tex
\chapter*{概要}
\thispagestyle{empty}
%
\begin{center}
  \scalebox{1.5}{}\\
\end{center}
\vspace{1.0zh}
%

本レポートでは,インテリジェントロボットモーション第11回「軌道計画と軌道生成」の解説を行う.
マニピュレータロボットや移動型ロボットなどをある地点から目標地点まで動かすときにニュートン・オイラー法やラグランジュ法
を用いて導出した運動学に基づいてマニピュレータが出すトルクや力を計算するが,このように
計算された力は各時刻の力を表すものであり,ロボットは軌道を描くことはできず,ロボットの通るべき経路は考慮されない.したがって,ロボットの
通るべき「軌道」を「計画」し,その軌道上をロボットが追従するように出すべき力を連続な力で表し,「軌道」を「生成」する必要がある.
はじめに軌道計画(経路計画)と軌道計画の違いと概念について解説し,次に軌道計画の主要なアルゴリズムの基礎的な理論について簡単に紹介する.最後にbang-bang制御による軌道生成の解説を行う.

キーワード: ロボット,経路計画,軌道生成
