%!TEX root = ../thesis.tex

\subsection{経路と軌道の違い}

軌道に対応する英単語として"Path"と"Trajectory"という単語が挙げられる.
この2つの英単語は運動の計画において違う意味を持ち,その意味は以下のようになる.
\begin{quote}
     \begin{itemize}
      \item Path:ロボットがどこを通るか
      \item Trajectory:ロボットがどのような運動で通るか
     \end{itemize}
\end{quote}
このように,"Path"と"Trajectory"には違いがあり,これらを日本語で再定義すると,"Path"は「経路」,
"Trajectory"は「軌道」とすることができる.そして,ロボットがどこを通るかを計画することを"Path planning",ロボットがどのような運動で通るかを決めることを"Trajectory generation"
と呼ぶ.以降,区別を明確にするために"Path planning"を「経路計画」,"Trajectory generation"を「軌道生成」と呼ぶことにする.
\newpage
